\documentclass[a4paper, 12pt]{article}
\usepackage[a4paper, top=2cm, bottom=3cm]{geometry}

\usepackage[T1,T2A]{fontenc}
\usepackage[utf8]{inputenc}
\usepackage[english,russian]{babel}

\usepackage[backend=biber, doi=false,isbn=false]{biblatex}
\addbibresource{references.bib}
\renewbibmacro{in:}{%
  \ifentrytype{article}{}{\printtext{\bibstring{in}\intitlepunct}}}

\usepackage{amsmath}
\usepackage{amssymb}
\usepackage{amsthm}
\usepackage{mathrsfs}
\usepackage{mathtools}
\usepackage{booktabs}
\usepackage[bottom]{footmisc}

\usepackage[
    colorlinks=true,
    allcolors=black,
    urlcolor=blue,
]{hyperref}

\newtheorem{theorem}{Теорема}[section]
\newtheorem{corollary}{Следствие}[theorem]
\newtheorem{lemma}[theorem]{Лемма}
\newtheorem{proposition}[theorem]{Утверждение}

\theoremstyle{definition}
\newtheorem{definition}[theorem]{Определение}
\newtheorem{example}[theorem]{Пример}
\newtheorem{remark}[theorem]{Замечание}

\begin{document}

\section{Вклад конической точки в формуле Лефшеца в терминах преобразования Фурье-Лапласа}

\subsection{Общие определения}

Рассмотрим некомпактное многообразие \({ \hat{M} }\), составленное из компактного основания \({ M }\) с границей \({ \Omega }\) и присоединённого к нему бесконечного цилиндра \({ [0, +\infty) \times \Omega }\). Ему соответствует многообразие \({ \mathcal M }\) с конической точкой \({ \alpha }\). Рассмотрим на \({ \hat{M} }\) соболевские пространства \({ H^{s}(\hat{M}) \coloneqq H^{s, 0}(\mathcal M) }\). (См. \cite[p. 46, Definition 2.3]{nazaikinskii2005elliptic}.)

Пусть \[
    D : H^{s}(\hat{M}) \to H^{s - m}(\hat{M}) \]
--- дифференциальный оператор порядка \({ m }\), совпадающий на \({ [0, +\infty) \times \Omega }\) с оператором \({ D_{\infty} \coloneqq D_{\infty}(-i \partial_{t}, \omega, -i \partial_{\omega}) }\), \[
    D_{\infty} : H^{s}(\mathbb R_{t} \times \Omega) \to H^{s-m}(\mathbb R_{t} \times \Omega)\,.
\]

При сопряжении \({ D_{\infty} }\) с преобразованием Фурье мы получаем, фактически, конормальный символ оператора \({ D }\): \[
    \sigma_{c}(D)(p) = \mathcal F D_{\infty} \mathcal F^{-1} = D_{\infty}(p, \omega, -i \partial_{\omega})\,, \quad p \in \mathbb C\,.
\] Нам удобно вместо преобразования Фурье рассмотреть преобразование Фурье-Лапласа (см. \cite{taubes1987gauge} и \cite{mrowka2016index}): \[
    \begin{gathered}
        \mathcal F_{z} : L^2(\mathbb R_{t} \times \Omega) \to L^2(\mathbb S^{1}_{t} \times \mathbb S^{1}_{z} \times \Omega)\,, \\
        \left( \mathcal F_{z}u \right)(t, z, \omega) \coloneqq z^{\frac{t}{2\pi}} \sum_{n} z^{n} u(t + 2\pi n, \omega)\,.
    \end{gathered}
\] Обратное преобразование тогда задаётся формулой \[
    u(t, \omega) = \frac{1}{2\pi i} \oint_{\lvert z \rvert = 1} z^{-\frac{t}{2\pi}} (\mathcal F_{z}u)(t, z, \omega)\: \frac{dz}{z}\,.
\] При этом преобразовании оператор \({ D_{\infty} }\) переходит в семейство операторов \[
    \begin{gathered}
        \widetilde D_{\infty}(\theta) \coloneqq \mathcal F_{z} D_{\infty} \mathcal F_{z}^{-1} = D_{\infty}(-i \partial_{t} - \frac{\theta}{2\pi}, \omega, -i \partial_{\omega})\,, \\
        \widetilde D_{\infty}(\theta) : H^{s}(\mathbb S^{1}_{t} \times \Omega) \to H^{s-m}(\mathbb S^{1}_{t} \times \Omega)\,,
    \end{gathered}
\] где для удобства \({ z \coloneqq e^{i\theta} }\), \({ \theta \in [0, 2\pi] }\), а \({ H^{s}(\mathbb S^{1}_{t} \times \Omega) }\) суть расслоения над \({ \mathbb S^{1}_{t} }\) с функцией склейки \({ e^{it} }\).

\begin{definition}
    Оператор \({ D }\) называется эллиптическим, если 
    \begin{enumerate}
        \item \({ \sigma(D) }\) обратим на \({ T^{*}\hat{M} }\) вне нулевого сечения;
        \item \({ \widetilde D_{\infty}(\theta) }\) обратим для любого \({ \theta \in [0, 2\pi] }\).
    \end{enumerate}
\end{definition}

Из \cite[Теорема 2]{рабинович1982алгебре} следует

\begin{theorem}
    Если \({ D }\) --- эллиптический, то он фредгольмов.
\end{theorem}
    
В дальнейшем нам так же понадобится понятие следа оператора в смысле сужения его на некоторое подмногообразие (см. \cite{новиков1966следы}.) Пусть \[
    \begin{gathered}
        i : \Omega \hookrightarrow \mathbb R \times \Omega\,, \quad i(\omega) = (0, \omega)\,, \\
        A : C^{\infty}(\mathbb R \times \Omega) \to C^{\infty}(\mathbb R \times \Omega)\,.
    \end{gathered}
\]

\begin{definition}
    Следом оператора \({ A }\) на \({ \Omega }\) называется композиция \[
        \tau(A) : C^{\infty}(\Omega) \overset{i_*}{\longrightarrow} \mathcal D'(\mathbb R \times \Omega) \overset{A}{\longrightarrow} \mathcal D'(\mathbb R \times \Omega) \overset{i^{*}}{\longrightarrow} C^{\infty}(\Omega)\,,
    \] где \[
    \begin{gathered}
        (i_*u)(t, \omega) = u(\omega) \delta(t)\,, \\
        (i^{*}u)(\omega) = u(0, \omega)\,.
    \end{gathered}
    \]
\end{definition}

\subsection{Вклад конической точки \({ \alpha }\) в формуле Лефшеца}

Рассмотрим комплекс из одного эллиптического оператора 
\begin{equation}
    \label{eq:the_complex}
    0 \to H^{s}(\hat{M}) \overset{D}{\longrightarrow} H^{s-m}(\hat{M}) \to 0\,.
\end{equation} 
Пусть \({ T_{j} }\), \({ j = 1, 2 }\), --- геометрический эндоморфизм комплекса (\ref{eq:the_complex}), заданный диффеоморфизмом \({ \hat{M} \to \hat{M} }\), совпадающим на \({ [0, +\infty) \times \Omega }\) с отображением \({ g : \mathbb R_{t} \times \Omega \to \mathbb R_{t} \times \Omega }\) сдвига по переменной \({ t }\) на некоторое число \({ \lambda \neq 0 }\).

\begin{proposition}
    \label{prop:conical_contribution}
    Вклад неподвижной точки \({ \alpha }\) в формуле Лефшеца выражается как \[
        \mathcal L_{\operatorname{sing}} = i \operatorname{Trace} \int_{0}^{2\pi} \tau\left( \widetilde T_{1}(\theta) \widetilde D_{\infty} ^{-1}(\theta) \frac{\partial \widetilde D_{\infty}}{\partial \theta} (\theta) \right)\: d\theta\,,
    \] где \({ \widetilde T_{1}(\theta) = e^{-i \frac{\lambda \theta}{2 \pi}} g^{*} }\).
\end{proposition}

Для доказательства обозначим за \({ B = B(-i \partial_{t}, \omega, -i \partial_{\omega}) }\) псевдодифференциальный оператор с конормальным символом \[
    \sigma_{c}(D)^{-1}(p) \frac{\partial \sigma_{c}(D)}{\partial p}(p)\,.
    \] Для краткости будем опускать зависимость от \({ \omega }\) и писать, например, \({ B(-i \partial_{t}) }\) вместо \({ B(-i \partial_{t}, \omega, -i \partial_{\omega}) }\). Тогда, согласно \cite[Theorem 11.3]{nazaikinskii2005elliptic} (см. так же \cite{nazaikinskii1999atiyah}), \[
    \mathcal L_{\operatorname{sing}} = \frac{1}{2 \pi i} \operatorname{Trace} \int\limits_{\mathclap{\mathbb R}} \sigma_{c}(T_1)(p) \, B_{\infty}(p)\: dp\,,
\] где интеграл понимается в смысле его регуляризации \[
    \left( \frac{-1}{i \lambda} \right)^{l} \int_{-\infty}^{+\infty} \sigma_{c}(T_1)(p) \, B_{\infty}^{(l)}(p)\: dp
\] для достаточно большого \({ l }\).

\begin{lemma}
    Имеем, что 
    \begin{multline*}
        \left( \frac{-1}{i \lambda} \right)^{l} \int_{-\infty}^{+\infty} \sigma_{c}(T_1)(p) \,  B_{\infty}^{(l)}(p)\: dp \\ 
        = \left( \frac{-1}{i \lambda} \right)^{l} \int_{0}^{2\pi} \tau\left( \, e^{-i \frac{\lambda\theta}{2\pi}} g^{*} \, B_{\infty}^{(l)}\left( -i \partial_{t} - \frac{\theta}{2\pi} \right) \right) d\theta\,,
    \end{multline*}
    где след \({ \tau }\) отвечает вложению \({ \Omega \hookrightarrow \mathbb S^{1}_{t} \times \Omega }\).
\end{lemma}

\begin{proof}
    Подставляя разложение \({ \delta(t) = \frac{1}{2\pi} \sum_{k} e^{ikt} }\) в определение следа, получаем 
    \begin{multline*}
        \tau\left( \, e^{-i \frac{\lambda\theta}{2\pi}} g^{*} \, B_{\infty}^{(l)}\left( -i \partial_{t} - \frac{\theta}{2\pi} \right) \right) u(\omega) 
        \\
        \begin{aligned}
            &= \left[ e^{-i \frac{\lambda \theta}{2 \pi}} g^{*} \, B_{\infty}^{(l)}\left( -i \partial_{t} - \frac{\theta}{2\pi} \right) u(\omega) \delta(t) \right]_{t=0} \\
            &= \left[ \frac{1}{2\pi} e^{-i \frac{\lambda \theta}{2 \pi}} \sum_{k} B_{\infty}^{(l)}\left( k - \frac{\theta}{2\pi} \right) u(\omega) e^{ik(t + \lambda)} \right]_{t = 0} \\
            &= \left[ \frac{1}{2\pi} \sum_{k} e^{i (k \lambda - \frac{\lambda \theta}{2 \pi})} B_{\infty}^{(l)} \left( k - \frac{\theta}{2\pi} \right) \right] u(\omega)\,,
        \end{aligned}
    \end{multline*}
    то есть \[
        \tau\left( \, e^{-i \frac{\lambda\theta}{2\pi}} g^{*} \, B_{\infty}^{(l)}\left( -i \partial_{t} - \frac{\theta}{2\pi} \right) \right)
        = \frac{1}{2\pi} \sum_{k} e^{i \lambda (k - \frac{\theta}{2 \pi})} B_{\infty}^{(l)} \left( k - \frac{\theta}{2\pi} \right)\,.
    \]
    Отсюда
    \begin{multline*}
        \int_{0}^{2\pi} \tau\left( \, e^{-i \frac{\lambda\theta}{2\pi}} g^{*} \, B_{\infty}^{(l)}\left( -i \partial_{t} - \frac{\theta}{2\pi} \right) \right) d\theta \\
        = \frac{1}{2 \pi} \int_{0}^{2\pi} \sum_{k} e^{i \lambda (k - \frac{\theta}{2 \pi})} B_{\infty}^{(l)} \left( k - \frac{\theta}{2\pi} \right)\: d\theta 
        = \int_{-\infty}^{+\infty} e^{i \lambda p} B_{\infty}^{(l)} (p)\: dp\,.
    \end{multline*}
    Остаётся заметить, что \({ \sigma_{c}(T_{1}) = e^{i \lambda p} }\).
\end{proof}

Таким образом, если положить \({ \widetilde T_1(\theta) = e^{-i \frac{\lambda \theta}{2 \pi}} g^{*} }\), то вклад в формуле Лефшеца принимает вид 
\begin{equation}
    \label{eq:new_requilarized_contribution}
    \mathcal L_{\operatorname{sing}} = \frac{1}{2\pi i} \operatorname{Trace} \left( \frac{2 \pi}{i \lambda} \right)^{l} \int_{0}^{2\pi} \tau \left( \widetilde T_1(\theta) \widetilde B_{\infty}^{(l)}(\theta) \right)\: d\theta\,,
\end{equation}
поскольку \[
    \widetilde B_{\infty}^{(l)}(\theta) 
    = \frac{\partial^{l}}{\partial \theta^{l}} \left( B_{\infty}\left( -i \partial_{t} - \frac{\theta}{2\pi} \right) \right)
    = \left( \frac{-1}{2\pi} \right)^{l} B_{\infty}^{(l)}\left( -i \partial_{t} - \frac{\theta}{2\pi} \right)\,.
\] Наконец, интегрируя по частям в выражении (\ref{eq:new_requilarized_contribution}), получаем \[
    \mathcal L_{\operatorname{sing}} = \frac{1}{2\pi i} \operatorname{Trace} \int_{0}^{2\pi} \tau \left( \widetilde T_{1}(\theta) \, \widetilde B_{\infty}(\theta) \right)\: d\theta\,,
\] что есть в точности равенство из Утверждения~(\ref{prop:conical_contribution}), поскольку \[
    \widetilde B_{\infty}(\theta) = -2\pi \widetilde D_{\infty} ^{-1}(\theta) \frac{\partial \widetilde D_{\infty}}{\partial \theta} (\theta)\,.
\] 

\begin{remark}
    Последнюю формулу можно так же переписать в терминах цилиндра \({ \mathbb R_{t} \times \Omega }\): \[
        \mathcal L_{\operatorname{sing}} = \frac{1}{2\pi i} \operatorname{Trace} \tau \left( T_{1\infty} D^{-1}_{\infty} [D_{\infty}, t] \right)\,,
    \] где след \({ \tau }\) отвечает вложению \({ \Omega \hookrightarrow \mathbb R_{t} \times \Omega\,, \ \omega \mapsto (0, \omega) }\).
\end{remark}

\begin{remark}
    Можно так же рассматривать и более общие диффеоморфизмы вида \[
        \begin{gathered}
            g : (t, \omega) \mapsto (t + \lambda(\omega), g_{\alpha}(\omega))\,,  \\
            \lambda : \Omega \to \mathbb R \setminus \left\{ 0 \right\}\,, \quad 
            g_{\alpha} : \Omega \to \Omega\,.
        \end{gathered}
    \] Утверждение~(\ref{prop:conical_contribution}) в таком случае остаётся в точности тем же, просто \({ \lambda }\) в выражении для \({ \widetilde T_1(\theta) }\) будет зависеть от \({ \omega }\).
\end{remark}

\printbibliography

\end{document}
